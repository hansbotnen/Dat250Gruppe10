\section{Introduction}
\label{sec:introduction}
 The reason for our interest in this Node.js is its supposed simplicity combined with its quality of being powerful. Our motivation is also based on the fact that Node.js is a popular technology used in many popular real world applications today. We have created a prototype web application to use as a good base for evaluating Node.js. We will be looking into how simple and powerful Node.js is, and also compare the technology and results of this prototype to the Java EE project.
\\\\
Node.js, created by Ryan Dahl, was first introduced in 2009 as a 
open-source, cross-platform JavaScript run-time environment, running on Linux and Mac OS X (Windows launch in 2011). Dahl was inspired to create Node.js after using file upload on an image hosting service called Flickr, where the browser could not see how much of the file had been uploaded and had to query the Web server. Dahl criticized the limitations of concurrent connections in Apache HTTP servers. He strongly desired a solution to this problem, thus creating Node.js. As we will see later in the report, Dahl solved this concurrent connection problem in a clever way by combining several other technologies like Google's V8 JavaScript engine, an event loop and a low level I/O API. \cite{wiki:nodejs}
\\\\
New web technologies surface now and then, and these technologies offer 
many different solutions to different problems. Many popular Websites like Netflix, PayPal, LinkedIn and many more are using Node.js as one of their main technologies, but what makes Node.js stand out from other technologies? In this project we have implemented an auction application based on our previous project in this course. Just like the previous project we are running a server directly from Node.js, and connect our application to a remote database. We have been basing our design of this application (mostly) on the design of the previous project, so the functionality is almost the same. We will try to show through our hands-on experience why Node.js is more preferable than many other web technologies today.
\\\\
The rest of the this report is organized as follows: 
\begin{itemize}
\item Section "Background": an overview about the software technology, its history, architecture and functionality. 

\item Section "Demonstrator Prototype": high-level view of the software technology in action, how we implemented the prototype and a presentation of the prototype.

\item Section "Test Environment and Experimental Results": describes the software in a test-bed environment and what experiments have been done.

\item Section "Conclusion": gives an overview over the conclusions about the software technology, the prototype and what results we have achieved in this project.
\end{itemize}